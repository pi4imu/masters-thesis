Микролинзирование является частным случаем гравитационного линзирования. Гравитационное линзирование - это явление отклонения света от прямолинейной траектории в поле тяготения массивных тел. Типичная ГЛ система состоит из источника, массивной линзы (которой может быть скопление галактик, галактика и даже отдельная звезда) и наблюдателя. Продольные расстояния в такой системе всегда много больше поперечных, поэтому можно пользоваться приближением плоских линз (как в геометрической оптике). 

Величина, которая характеризует масштабы ГЛ и является основной шкалой поперечных расстояний - радиус Эйнштейна. Видно, что он зависит от массы линзы и от продольных расстояний в системе. Расстояния в системе понимаются в смысле углового расстояния по диаметру. Они обратно пропорциональны постоянной Хаббла.

Важной особенностью ГЛ является возможность формирования нескольких изображений одного и того же источника. Это связано с тем, что между вылетом фотонов из источника и приходом их к наблюдателю проходит некоторое время, которое состоит из двух слагаемых: геометрического (связана просто с тем, что луч света распространяется не по прямолинейной траектории) и гравитационного (в соответствии с ОТО время около гравитирующих тел идёт медленнее). Соответственно, если мы наблюдаем одну и ту же кривую блеска в разных изображениях, но пришедшую к нам в разное время, то, измеряя временную задержку между изображениями, мы можем измерить постоянную Хаббла, так как множитель в формуле для времени обратно пропорциональны ей.

Для таких измерений блеск источника должен быть переменным - в таком случае разные изображения будут изменять свой блеск так же, как и источник, но с какой-то временной задержкой. Одним из возможных классов источников являются квазары - они очень яркие и почти точечные. Другим подходящим классом источников являются сверхновые - их кривые блеска имеют чётко выраженный пик, а наблюдения занимают сравнительно небольшие времена, что упрощает измерения. На текущий момент известно только две ГЛ сверхновые - SN Refsdal, открытая в 2014 году (которая названа в честь астронома Сьюра Рефсдала, который первым предложил использовать ГЛ сверхновые в 1964 году), и ещё одна. Однако с запуском в ближайшее время  обзора LSST (Legacy Survey of Space and Time) в Обсерватории имени Веры Рубин  ЛССТ (ее теперь называют “обсерватория имени Веры Рубин”) ожидается открытие десятков таких систем, что делает задачу разработки алгоритмов анализа ГЛ сверхновых важной и своевременной.
Кроме того, для сверхновых типа Ia - стандартных свечей - можно оценить болометрическую (интегральную по всему спектру) светимость из независимых соображений, а значит, получить абсолютное значение усиления светового потока. Эта информация, недоступная для линзированных квазаров, позволяет уменьшить количество свободных параметров модели линзы и снять определенные вырождения (например, mass-sheet degeneracy), а значит, уменьшить ошибки на  оцениваемые космологические параметры.  (то, что курсивом - необязательно -- отличная фраза, оставь)

Дальнейший рассказ связан непосредственно с SN Refsdal, вкратце опишу конфигурацию её изображений. Сверхновая находится на красном смещении z = 1.5 в рукаве спиральной галактики, которая линзируется скоплением галактик как целым таким образом, что возникают три изображения спиральной галактики - 1.1, 1.2 и 1.3. При этом в изображении 1.1 сверхновая дополнительно линзируется эллиптической галактикой в скоплении таким образом, что формируются четыре её изображения S1-S4, расположенных в виде “креста Эйнштейна”. Изображение SX (1.2) интересно тем, что его появление и яркость были предсказаны с хорошей точностью, а изображение SY, согласно теоретическим оценкам,  давно потухло и не наблюдалось.

Теперь, собственно, о микролинзировании. Это гравитационное линзирование на отдельных звёздах. Его масштабы - характерные углы отклонения света - в миллион раз меньше, чем при обычном линзировании на галактиках и скоплениях. На текущий момент разрешить микро-изображения не представляется возможным, но их можно “засечь” по кратковременному увеличению яркости, как это проиллюстрировано на картинке.... Что тут происходит.

Для моделирования влияния микролинзирования на кривые блеска использовалась программа микроленс, которая моделирует карты микрокаустик в плоскости источника, основываясь на распределении звёзд в плоскости линзы. Светлые области означают, что ТОЧЕЧНЫЙ источник, находясь в них, усиливается, тёмные - что ослабляется. Каустики - линии, которые эти области разделяют. Основными параметрами карты являются уже ранее упоминавшиеся поверхностные плотности звёзд и непрерывно распределённой тёмной материи в плоскости линзы-галактики, так называемый внешний сдвиг, учитывающий вклад гравитационного потенциала скопления, а также функция масс звёзд (впрочем, для простоты в дальнейшем предполагается, что все звёзды имеют одинаковую массу, равную солнечной). Карты выглядят вот так вот…, то есть чем больше звёзд-линз, тем богаче сеть каустик. 

Теперь посмотрим на то, как микролинзирование влияет на кривые блеска. Сверхновая моделируется расширяющимся кружком с постоянной поверхностной яркостью. Для иллюстрации мы выбрали два различных начальных положения сверхновой. Видно, что если сверхновая расширяется всё время в области постоянного усиления, то и кривая блеска практически не изменяется (вклад оранжевым цветом), но если в своём расширении кружок пересекает каустики, то есть области сильного усиления, то форма кривой блеска (синяя) может заметно исказиться - флуктуации порядка 0.5 звёздных величин.

Для источников, размеры которых сравнимы по порядку или меньше радиуса Эйнштейна данной системы, существенно учитывать микролинзирование. Мы оценили максимальный угловой размер SN Refsdal (из модели зависимости радиуса фотосферы от времени) - он составляет примерно 4% от радиуса Эйнштейна, который мы оценили по расстояниям из красных смещений линзы и сверхновой. Значит, поскольку угловой радиус сверхновой всё время её расширения меньше радиуса Эйнштейна, микролинзирование может вносить существенные искажения в её кривые блеска.

Мы построили карты усилений для изображений S1-S4 SN Refsdal, но вычисления проводили только для пары изображений S1 и S2, так как временная задержка между ними рассчитана с наименьшими погрешностями. Для того, чтобы профиль яркости сверхновой можно было задать на большом количестве точек вдоль радиуса, необходимы карты с высоким разрешением (много пикселей в 1 р.Э.). Построены карты микролинзирования для областей изображений S1 и S2 в масштабе 1 радиус Эйнштейна = 2000 пикселей. Проведено небольшое статистическое исследование этих карт. Гистограммы по их значениям сравниваются с гистограммами карт с аналогичными параметрами из обзора GERLUMPH, полученные независимой командой исследователей и которые лежат в открытом доступе. Кроме того, на гистограммы нанесена зависимость \mu^{-3} – такой зависимости подчиняется усиление точечных источников для больших \mu. (см. Wambsganns 1992). Максимальный радиус SN Refsdal на этих картах – 72 пикселя. 

Затем на каждую из карт «бросаем» по 200 расширяющихся сверхновых. Для каждого положения источника считаем флуктуации микролинзирования сразу и от плоского источника, и от источника с гауссовым профилем яркости. Сначала на карту для S1, затем на карту для S2. Не очень хорошо видно, но для обеих карт в редких случаях гауссовый источник даёт более сглаженные флуктуации, в основном микролинзирования вообще нет, так как источник во время расширения не пересекает каустики).

Рассматриваются кривые блеска SN Refsdal, наблюдаемые в изображениях S1 и S2. Временная задержка и относительное усиление взяты из статьи (Baklanov et al. 2021): dt_true = 9.5 days, dm_true = - 0.14 (\mu_{21} = 1.14). На основе построенной физической модели предсверхновой, удовлетворяющей фотометрическим наблюдениям в разных фильтрах, получены уточнённые значения временных запаздываний и коэффициентов усиления между изображениями SN Refsdal.

Затем к кривым блеска в различных изображениях прибавляются уникальные для каждого изображения флуктуации. К каждой из кривых блеска можно добавить по 50 флуктуаций для каждого изображения, всего получается 50 х 50 = 2500 реализаций. Одна из таких реализаций. Видно, что в этом случае форма кривой блеска в изображении S1 в первые 150 дней сильно искажается, а вот кривая в изображении S2 всего лишь сдвигается, испытывая одинаковое смещение во всех точках.  

Теперь при помощи emcee фиттируем кривую блеска в изображении S1 кривой в изображении S2. Свободные параметры: dt, dm. Из всех фитов выбираем тот, для которого функция правдоподобия максимальна: Непосредственно для этой реализации получается: best_dt = - 8.6 days, best_dm = -0.84.

Та же самая операция производится для всех возможных комбинаций отдельно для плоского источника и отдельно для гауссового. После чего по полученным временным задержкам и относительным усилениям строятся гистограммы. 

Результаты:
(а) на наших картах источник много раз попадал в области без каустик, из-за чего для большого количества реализаций микролинзирование вообще никак не сказалось, отсюда такой высокий и узкий пик около истинного значения dt,
(б) тем не менее, по dm разброс гораздо шире, что означает, что в реальности затруднительно отделить макроусиление от линзы от микроусиления от отдельных звезд. 

Предварительный вывод: в идеальных условиях, когда наблюдения производятся часто и при помощи идеальных телескопов,  вклад микролинзирования в значения временной задержки и относительного усиления по крайней мере между изображениями S1 и S2 невелик (ошибки из-за микролинзирования сравнимы с ошибками самой величины). Цель была оценить возможный вклад микролинзирования при условии проведения идеального эксперимента. То есть мы наблюдаем систему практически непрерывно, ошибки наблюдений сведены к 0. Можно также добавить, что планируется проведение более приближенного к реальности эксперимента, когда учитывается реалистичный каденс (то есть частота наблюдений) и реалистичные характеристики телескопов.

И над слайдом с заключением надо еще поработать.
1) Промоделирован вклад микролинзирования в кривые блеска SN Refsdal в модели сверхновой с постоянным профилем яркости — это ок
2) Получена оценка погрешности, вносимой ТОЛЬКО микролинзированием при условии проведения идеального эксперимента в определение временных задержек между изображениями S1-S2 SN Refsdal
— и быть готовым к вопросу, почему только S1-S2 пара рассматривается. В других парах микролинзирование тоже же есть
3) Показано, что даже в условиях идеального эксперимента точность измерения усилений составляет ... . Разброс вносится разветвленной сетью каустик, образованной звездами в галактике
4) Показано, что точность измерения временных задержек при наличии микролинзирования составляет порядка нескольких дней. Таким образом, для определения постоянной Хаббла по ГЛ необходимо использовать ГЛ системы с временными задержками, составляющими десятки дней. В случае с ГЛ Рефсдал это означает, что H0 наиболее надежно определяется по паре изображений S2-SX, а анализ изображений S1-S4 позволяет лучше ограничить модель линзы.

Открытые вопросы:
(а) Будет ли вклад микролинзирования плоского источника отличаться от гауссового? Для этого нужно знать, какой профиль яркости у SN Refsdal.
(б) Будут ли ошибки такими же маленькими, если кривая блеска будет наблюдаться после пика или в другом фильтре? В идеале стоит проверить и то, и то, но хочется понимать, корректно ли работает данный алгоритм.
(в) Есть основания полагать, что dt и dm должны быть скореллированы (например, как в (Kelly et al. 2016). Но нужно ли это учитывать при фиттировании?
