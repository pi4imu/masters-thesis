\noindent Идея такая: фиттируем кривую блеска в S1 кривой блеска в S2, сдвигая последнюю по времени и магнитуде (можно в другую сторону, разницы нет). \\ \\
$m_{S1}^{fit} = m_{S2}(t - \Delta t)-\Delta m$ \\

$\log p = -\frac{1}{2} \sum \big( m_{S1} - m_{S1}^{fit} \big)^2$ \rightarrow \max \\ \\

\noindent Подразумевается суммирование по всем точкам на кривой блеска. У меня не учтена модельная погрешность $\sigma_m$ (точнее, она предполагается постоянной). Можно модифицировать функцию правдоподобия следующим образом, но тогда нужно будет заново произвести все расчёты. \\ \\

$\log p = -\frac{1}{2} \sum \frac{( m_{S1} - m_{S1}^{fit} )^2}{\sigma_m^2} + \log \big( 2\pi\sigma_m^2 \big)$ \\ \\

\noindent Из всех кривых $m_{S1}^{fit}$ выбирается та, для которой функция правдоподобия максимальна (это и есть Best fit на рисунке слева), значения $\Delta t$ и $\Delta m$ для неё запоминаются, описанные выше действия повторяются для других реализаций. \\

$\Delta t_{true} = 9.5$ days, 
$\Delta m_{true} = -0.14$ \\
$\Delta t_{best} = -8.6$ days,
$\Delta m_{best} = -0.84$ 


\newpage