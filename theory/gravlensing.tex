%\subsection{Гравитационно-линзированная система}

\nl{Лучше начать как-нибудь по-другому, чтобы диплом читался как единый текст. В духе: Гравитационное линзирование - это ... Выделяют два режима  - сильный и слабый. Описание. Здесь и далее речь пойдет об эффекте сильного линзирования. } В этом разделе приведены основные понятия, связанные с описанием так называемого \textit{сильного} гравитационного линзирования. Большая часть информации в данном разделе подробно представлена в книге  (\cite{gravlensbook}). В рамках общей теории относительности можно показать, что в гравитационном поле точечного массивного тела \textit{(линзы)} с ньютоновским потенциалом $\Phi$ угол отклонения светового луча равен

\begin{equation}
\hat{\alpha}=\frac{2}{c^{2}} \int_{-\infty}^{+\infty} \vec{\nabla}_{\perp} \Phi \ \mathrm{d} z,
\end{equation}

где $z$ - координата вдоль невозмущённой траектории распространения света (\cite{gl_all}). При этом предполагается, что $\Phi/c^2 \ll 1$, что выполняется практически всегда: например, для самых массивно гравитационно связанных объектов во Вселенной -- скоплений галактик --  $|\Phi|/c^2 \sim 10^{-4} \ll 1$. Это же условие обеспечивает малость углов: при сильном линзировании углы отклонения света порядка угловых секунд. Из того, что $\Phi$ не зависит от массы тела, попавшего в гравитационное поле, следует, что все фотоны отклоняются на один и тот же угол. Таким образом, сильное гравитационное линзирование ахроматично, то есть не зависит от длины волны. 
%Для точечной линзы $\Phi=-\frac{G M}{r}$, где $M$ - её масса, а значит, что углы отклонения массивом таких линз могут быть сложены по принципу суперпозиции. 

Непосредственно преломление происходит на очень коротком участке траетории света. В реалистичных моделях линз, в которых масса, вызывающая линзирование, распределена трёхмерным образом, используется приближение плоских линз по аналогии с геометрической оптикой. Это всегда оправдано: характерные размеры самого большого объекта, который может быть линзой, - скопления галактик - порядка 1 Мпк, в то время как продольные расстояния между объектами системы порядка 100-1000 Мпк. Типичная гравитационно-линзированная система изображена на Рисунке \ref{fig:gravlensfig}. Углами $\boldsymbol{\beta}$ и $\boldsymbol{\theta}$ задаются положения источника света и его изображения соответственно. Угол отклонения светового луча $\alpha$ по порядку величины составляет 1 угловую секунду при сильном линзировании на галактике.

\begin{figure}[H]
    \centering
	\includegraphics[scale=1.0]{pics/Gravitational_Lensing_Strong_Weak_and_Micro.eps}
	\caption{Типичная гравитационно линзированная система (\cite{gravlensbook}). Здесь $\beta$ и $\theta$ - углы между оптической осью (пунктирная линия) и источником и его изображением соответственно, $\alpha$ - угол отклонения светового луча, $\eta$ и $\xi$ - расстояния от оптической оси до источника и его изображения соответственно,  $D_d$ - расстояние между наблюдателем и плоскостью линзы, $D_{ds}$ - между плоскостями линзы и источника, $D_s$ - между наблюдателем и источником. Следует отметить, что в общем случае, $D_s \neq D_d + D_{ds}$. \label{fig:gravlensfig} }
   \end{figure} 
Уравнение линзы (соотношение между положениями источника, его изображения и углом преломления): - ПРОВЕРИТЬ

\begin{equation}\label{nablapsi}
\boldsymbol{\beta} = \boldsymbol{\theta} -  \hat{\alpha}(\boldsymbol{\theta}), \textrm{где} \  \hat{\alpha}(\boldsymbol{\theta}) = \frac{D_S}{D_{DS}}\nabla \boldsymbol{\theta}
\end{equation}



Можно показать, что

\begin{equation}\label{nabla2psi}
\kappa({\boldsymbol{\theta}}) = \frac{1}{2}\nabla^2 \Psi(\boldsymbol{\theta}),
\end{equation}

где $\kappa$  - безразмерная поверхностная плотность масс в линзе  (\textit{convergence}), $\Psi(\boldsymbol{\theta})$ - линзирующий гравитационный потенциал. Пусть масса линзы распределена в пространстве по закону $\rho(D_d \boldsymbol{\theta}, z)$, где $z$ - координата вдоль оптической оси. \textit{Поверхностная плотность} линзы задаётся следующим соотношением:

\begin{equation}\label{sigmasurf}
\Sigma(D_d \boldsymbol{\theta})=\int \rho(D_d \boldsymbol{\theta}, z) \mathrm{d} z
\end{equation}

С учётом этого для $\kappa$ также верно соотношение

\begin{equation}\label{convergence}
\kappa = \frac{\Sigma(D_d \boldsymbol{\theta})}{\Sigma_{crit}}, \ \ \ \ \  \textrm{где} \ \ \ \ \ \ \Sigma_{crit}=\frac{c^{2}}{4 \pi G} \frac{D_{\mathrm{s}}}{D_{\mathrm{d}} D_{\mathrm{ds}}},
\end{equation}

($c$ - скорость света, $G$ - гравитационная постоянная, $D_d$, $D_s$ и $D_{ds}$ - расстояния от наблюдателя до плоскости линзы, до источника и между плоскостями линзы и источника соответственно (см. Рис. \ref{fig:gravlensfig}).) По порядку величины $\Sigma_{crit} \sim 1 \ \textrm{г/см}^2$. Радиус такой окружности в плоскости линзы, плотность внутри которой равна критической ($\kappa =  1$), называется \textit{радиусом Эйнштейна}. Обычно он выражается в угловых единицах: 

\begin{equation}\label{r_ein}
\theta_{E}=\sqrt{\frac{4 G M}{c^{2}} \frac{D_{d s}}{D_{d} D_{s}}},
\end{equation}
где $M$ - масса линзы. Эта величина характеризует масштабы гравитационного линзирования и является основной шкалой расстояний при описании этого явления.

%\subsection{Усиление и искажение изображений}

В терминах линейной алгебры, гравитационное линзирование - это отображение плоскости источника на плоскость линзы, которое задаётся следующей матрицей:

\begin{equation}
\mathcal{A}(\boldsymbol{\theta})=\frac{\partial \boldsymbol{\beta}}{\partial \boldsymbol{\theta}}=\left(\delta_{i j}-\frac{\partial^{2} \psi(\boldsymbol{\theta})}{\partial \theta_{i} \partial \theta_{j}}\right)=\left(\begin{array}{cc}{1-\kappa-\gamma_{1}} & {-\gamma_{2}} \\ {-\gamma_{2}} & {1-\kappa+\gamma_{1}}\end{array}\right)
\end{equation}

Безразмерная плотность  $\kappa$ \textit{(convergence)}  отвечает за изотропное изменение линейных размеров изображения (увеличение или уменьшение),  двухкомпонентный сдвиг \textit{(shear)} $\gamma=(\gamma_1,\gamma_2)$, характеризует анизотропное искажение формы изображений. \textcolor{green}{КАРТИНКУ} Усиление изображения обратно пропорционально определителю этой матрицы (то есть якобиану этого отображения):

\begin{equation}
\mu=\frac{1}{\operatorname{det} \mathcal{A}} =  \frac{1}{(1-\kappa)^2-|\gamma|^2}
\end{equation}

Возможно существование некоторого множества точек, для которых выполняется соотношение $\operatorname{det} \mathcal{A}=0$, в которых усиление формально бесконечно \nl{добавить footnote, что в реальности бесконечных усилений не наблюдается, так как...}. Кривая, образуемая этими точками, называется каустикой, а её образ в плоскости линзы – критической кривой. 



%На Рисунке \ref{fig:caustics} видно, как ведёт себя изображение компактного источника при пересечении им складки \textit{(fold)} или излома \textit{(cusp)} каустики.
 
%\begin{figure}[H]
%    \centering
%	\includegraphics[scale=0.8]{pics/caust_intro.png}
%	\caption{Поведение изображения компактного источника при пересечении им излома (левая панель) или складки (правая панель) каустики  (\cite{narbart}). Также для иллюстрации изменения масштабов изображения нарисованы критические кривые в плоскости линзы. \label{fig:caustics}}
%\end{figure}


