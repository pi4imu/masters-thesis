Для каждого изображения вклад микролинзирования уникален и не зависит от других изображений, что вносит некоторую неопределенность в определение временных задержек между изображениями. Для количественных оценок точности определения $\Delta t$ между двумя кривыми блеска с учетом микролинзирования используется подход, детально описанный в работе \cite{doblerkeeton2006}. За основу была взята одна из кривых блеска SN Refsdal, рассчитанных в гидродинамической модели в различных частотных фильтрах для 400 дней с момента её взрыва (Бакланов и др., в подготовке). Она изображена на Рисунке \ref{fig:lightcurves}. 

\begin{figure}[H]
    \centering
	\includegraphics[scale=0.60]{pics/lightcurves.png}
	\caption{Кривые блеска SN Refsdal в различных фильтрах, полученные в гидродинамической модели (Бакланов и др., в подготовке). \label{fig:lightcurves}} 
\end{figure}

Для простоты анализа рассматривается кривая блеска только в одном фильтре -- F160W. Для иллюстрации влияния микролинзирования используются карты для областей изображений S1 и S2 (см. Рис. \ref{fig:s1s4}). SN Refsdal моделируется кругом с постоянной поверхностной яркостью, случайно расположенным на карте, расширяющимся с постоянной скоростью 5000 км/с. Выбор такого значения скорости в некоторой степени произволен для лучшей иллюстрации. Предполагается, что выбранная выше кривая блеска наблюдается в изображении S1. Для имитации изображения S2 имеющаяся кривая блеска дублируется и сдвигается по времени (далее эта разница обозначается как истинная временная задержка $\Delta t_{\textrm{ист.}}$) и звёздной величине. После этого к каждой из кривых блеска добавляются “шумы”, вызванные микролинзированием, полученные при помощи программного пакета {\tt{SNTD}} на соответствующих изображению картах. Для “зашумленных” кривых блеска временная задержка между изображениями определяется путем минимизации следующего функционала (аналогично работе \cite{doblerkeeton2006}):

\begin{equation}\label{chi2}
\chi^{2}(\Delta t)=\frac{1}{N} \sum_{i=1}^{N} \frac{1}{\sigma_{i}^{2}}\left[D^{+}\left(t_{i}\right)-D^{-}\left(t_{i}-\Delta t\right)-k\right]^{2},
\end{equation}
где $D^+$ и $D^-$ -- значения кривых блеска, выраженные в звёздных величинах, $\sigma_i = 0.05$ звёздных величин -- фотометрическая погрешность (она предполагается постоянной во времени), $k$ -- нормировочная постоянная, связанная с различным макроусилением "изображений". Далее эта процедура многократно повторяется, после чего по полученным значениям разброса временных задержек находится стандарное отклонение соответствующего распределения $\Delta t$ по формуле
\begin{equation}\label{sigmadeltat}
\sigma_{\Delta t}=\sqrt{\frac{1}{N} \sum_{i=1}^{N}\left((\Delta t)_{i}-\overline{\Delta t}\right)^{2}},
\end{equation}
которую можно трактовать как показатель неопределённости, вызванной микролинзированием, во временной задержке между изображениями. Результаты представлены в следующей секции.
