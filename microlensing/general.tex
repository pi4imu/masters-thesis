При детальном изучении временной задержки нельзя пренебречь линзированием на отдельных звёздах (или других компактных объектах) в галактике-линзе. Этот эффект называется \textit{микролинзированием}. Его масштабы (то есть характерные углы отклонения света) в миллион раз меньше масштабов сильного линзирования. На текущий момент разрешить микро-изображения (множественные изображения, возникающие в результате микролинзирования) не представляется возможным. Однако вполне возможно “засечь” микролинзирование благодаря кратковременному увеличению или ослаблению яркости источника. 

Известно, что микролинзирование при больших размерах источника не вносит существенного вклада в блеск источника (\cite{refsdalstabell1991}).

Микролинзирование может вносить существенный вклад в наблюдаемую кривую блеска, если размер источника не превышает радиус Эйнштейна $\theta_E$ для звезды-микролинзы. 

Напомним, что эта величина задаётся соотношением \eqref{r_ein}

$$\theta_{E}=\sqrt{\frac{4 G M}{c^{2}} \frac{D_{d s}}{D_{d} D_{s}}},$$

Для конфигурации SN Refsdal (см. Рис. \ref{fig:snrefsdalfig}) оценим $\theta_E$ для звезды с массой $1 M_{\odot}$. Расстояния до галактики-линзы (члена скопления MACSJ1149.6+2223, $z_L=0.54$), до источника ($z_S=1.49$), а также между линзой и источником равны, согласно формуле \eqref{ang_dia_dist}, 
$$ D_d=D_A(0,z_L)=1311.54 \ \textrm{МПк}, $$
$$ D_s=D_A(0,z_S)=1744.81 \ \textrm{МПк}, $$
$$ D_{ds}=D_A(z_L,z_S)=932.47 \ \textrm{МПк}, $$
соответственно. В результате, радиус Эйнштейна для звезды с массой $1 M_{\odot}$ составляет, согласно формуле \eqref{r_ein}, $$\theta_E \approx 1.8 \cdot 10^{-6} \textrm{угловых секунд.} $$ 

Для сверхновых II типа максимальный размер фотосферы составляет $R_{SN} \sim 10^{15}$ см (\cite{razmer}). По результатам гидродинамического моделирования расширения оболочки SN Refsdal её максимальный размер составляет $R_{SN} = 1.75\cdot10^{15}$ см (\cite{petrnat2020}), или, в угловых единицах, $$\theta_{SN} = \frac{R_{SN}}{D_S} \sim 6.7 \cdot 10^{-8}$$ угловых секунд. Таким образом, угловой размер SN Refsdal не превышает характерный радиус Эйнштейна: $$ \theta_{SN} = 0.037 \cdot \theta_E$$а значит, микролинзирование может вносить существенные искажения в её кривые блеска. %Эта оценка пригодится нам немного позже для оценки масштаба карт.