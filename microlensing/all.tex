Для того, чтобы профиль яркости сверхновой
можно было задать на большом количестве
точек вдоль радиуса, необходимы карты с
высоким разрешением (много пикселей в 1 р.Э.)

Построены карты микролинзирования для областей изображений S1 и S2 в масштабе 1 радиус Эйнштейна = 2000 пикселей

Проведено небольшое статистическое исследование этих карт. Гистограммы по их значениям сравниваются с гистограммами карт GERLUMPH  \cite{gerlumph}
Кроме того, на гистограммы нанесена зависимость \mu^{-2} – такой зависимости дчиняется усиление точечных источников для больших \mu 
(см. \cite{wambsganss1992-microlens})

Максимальный радиус SN Refsdal на этих картах – 72 пикселя. Есть спорное мнение: если радиус в пикселях будет сильно меньше, то флуктуации микролинзирования от плоского и гауссового источников будут практически неразличимы


протяженный источник:

https://arxiv.org/pdf/astro-ph/0608391.pdf стр 4 слева снизу

https://arxiv.org/pdf/1708.00003.pdf уравнение 9

https://arxiv.org/pdf/1805.04525.pdf раздел 3

https://arxiv.org/pdf/1903.00510.pdf уранение 15

https://arxiv.org/pdf/2002.08378.pdf раздел 2.2

Затем на каждую из карт «бросаем» по 50 расширяющихся сверхновых. Для одного положения считаем флуктуации микролинзирования сразу и от плоского источника, и от источника с гауссовым профилем яркости. Сначала на карту для S1… 

…затем на карту для S2. Не очень хорошо видно, но для обеих карт в редких случаях гауссовый источник даёт более сглаженные флуктуации, в основном микролинзирования вообще нет, так как источник во время расширения не пересекает каустики)

Слева: рассматриваются кривые блеска SN Refsdal, наблюдаемые в изображениях S1 и S2. Временная задержка и относительное усиление взяты из статьи (Baklanov et al. 2021):
dt_true = 9.5 days
dm_true = - 0.14
(\mu_{21} = 1.14)

Затем к кривым блеска в различных изображениях прибавляются уникальные для каждого изображения флуктуации (отдельно для плоского источника, отдельно для гауссового). К каждой из кривых блеска можно добавить по 50 флуктуаций для каждого изображения, всего получается 50 х 50 = 2500 реализаций.
Справа: одна из таких реализаций. Видно, что в этом случае форма кривой блеска в изображении S1 в первые 150 дней сильно искажается, а вот кривая в изображении S2 всего лишь сдвигается, испытывая одинаковое смещение во всех точках.  

Теперь при помощи emcee фиттируем кривую блеска в изображении S1 кривой в изображении S2. Свободные параметры: dt, dm. 

Из всех фитов выбираем тот, для которого функция правдоподобия максимальна:

Непосредственно для этой реализации получается:
best_dt = - 8.6 days
best_dm = -0.84
Та же самая операция производится для всех возможных комбинаций (всего 2500 раз) отдельно для плоского источника и отдельно для гауссового. После чего по полученным временным задержкам и относительным усилениям строятся гистограммы. 

(а) на наших картах источник много раз попадал в области без каустик, из-за чего для большого количества реализаций микролинзирование вообще никак не сказалось, отсюда такой высокий и узкий пик около истинного значения dt, 
(б) тем не менее, по dm разброс гораздо шире, как это интерпретировать – пока непонятно. 
(в) Из гистограмм видно, что принципиальной разницы между плоским и гаусовым источником нет, но это не значит, что это так на самом деле. 
(г) Фит не предполагает корелляции между dt и dm, поэтому распределение вытянуто вертикально.

Теперь увеличим количество различных положений сверхновой на карте: по 200 на каждой из них, всего 200х200=40000 реализаций. Но теперь только для плоского источника.

Гистограмма по 40000 точек не очень сильно отличается от гистограммы для 2500 точек. Возможно, для полноценного статичестического исследования не нужно так много реализаций.

Предварительный вывод: в идеальных условиях вклад микролинзирования в значения временной задержки и относительного усиления по крайней мере между изображениями S1 и S2 невелик (ошибки из-за микролинзирования сравнимы с ошибками самой величины)
Открытые вопросы: 
(а) Будет ли вклад микролинзирования  плоского источника отличаться от гауссового? Для этого нужно знать, какой профиль яркости у SN Refsdal.
(б) Будут ли ошибки такими же маленькими, если кривая блеска будет наблюдаться после пика или в другом фильтре? В идеале стоит проверить и то, и то, но хочется понимать, корректно ли работает данный алгоритм.
(в) Есть основания полагать, что dt и dm должны быть скореллированы (например, как в (Kelly et al. 2016). Но нужно ли это учитывать при фиттировании?



Промоделирован вклад микролинзирования в кривые блеска SN Refsdal в модели сверхновой с постоянным профилем яркости
Получена оценка погрешности, вносимой в определение временных задержек между изображениями S1 и S2 SN Refsdal только микролинзированием в условиях идеального эксперимента: 1-3 дня
Показано, что даже в условиях идеального эксперимента точность измерения усилений составляет порядка 1 зв.вел. 


