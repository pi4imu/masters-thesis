Для определения постоянной Хаббла необходимы не только измерения временных задержек между различными изображениями, но и информация о гравитационном потенциале скопления, играющего роль линзы. На текущий момент представлен ряд моделей распределения вещества в скоплении галактик MACS J1149+2223, выступающего в роли линзы, разработанных разными коллективами авторов, и приведены теоретические предсказания временных задержек $\Delta t$ и коэффициентов усиления $\mu$ изображений S2, S3, S4, SX и SY относительно изображения S1 (\cite{treu2016}). Изображение SY далее не учитывается, так как оно не наблюдалось. В данной работе рассматриваются модели ‘Gri-g’, ‘Ogu-g’, ’Ogu-a’, ’Sha-g’, and ’Sha-a’, модели 'Die-a' и 'Zit-g' не учитываются\footnote{Названия моделей образованы первыми буквами фамилий руководителей команд, их разработавших: \textbf{Gri}llo, \textbf{Ogu}ri, \textbf{Sha}ron, \textbf{Die}go и \textbf{Zit}rin.}. Данные моделей подробно описаны в публикациях \cite{model_die}, \cite{model_gri}, \cite{model_ogu}, \cite{model_sha}, \cite{model_zit}. Предполагается, что $\Delta t$ и $\mu$ распределены нормально и не скоррелированы друг с другом, их средние значения и статистические погрешности приведены в Таблице \ref{tab:dtdm}.

\begin{table}[H]
 \caption{Рассматриваемые в этой работе временные задержки (в днях) и относительные усиления между изображениями SN Refsdal, предсказанные различными моделями распределения вещества \mbox{в скоплении галактик MACS J1149+2223 (\cite{treu2016}, Табл. 6).}}
 \label{tab:dtdm}
 \centering
 \scalebox{0.86}{
 \begin{tabular}{ | c | c | c | c | c | c | c | c | c |}
    \hline
    Модель & $\Delta t_{21}$ & $\Delta t_{31}$ & $\Delta t_{41}$ & $\Delta t_{X1}$ & $\mu_{21}$ & $\mu_{31}$ & $\mu_{41}$ & $\mu_{X1}$ \\ \hline
    Gri-g & $10.6^{+6.2}_{-3.0}$ & $4.8^{+3.2}_{-1.8}$ & $25.9^{+8.1}_{-4.3}$ & $361^{+19}_{-27}$ & $0.92^{+0.43}_{-0.52}$ & $0.99^{+0.52}_{-0.33}$ & $0.24^{+0.19}_{-0.20}$ & $0.36^{+0.11}_{-0.09}$ \\ \hline
    Ogu-g & $8.7 \pm 0.7$ & $5.1 \pm 0.5$ & $18.8 \pm 1.7$ & $311 \pm 24$ & $1.14 \pm 0.24$ & $1.22 \pm 0.24$ & $0.67 \pm 0.17$ & $0.27 \pm 0.05$ \\ \hline
    Ogu-a & $9.4 \pm 1.1$ & $5.6 \pm 0.5$ & $20.9 \pm 2.0$ & $336 \pm 21$ & $1.15 \pm 0.17$ & $1.19 \pm 0.17$ & $0.64 \pm 0.11$ & $0.27 \pm 0.03$ \\ \hline
    Sha-g & $6^{+6}_{-5}$ & $-1^{+7}_{-5}$ & $12^{+3}_{-3}$ & $277^{+11}_{-21}$ & $0.84^{+0.18}_{-0.06}$ & $1.68^{+0.55}_{-0.21}$ & $0.57^{+0.11}_{-0.04}$ & $0.25^{+0.05}_{-0.02}$ \\ \hline
    Sha-a & $8^{+7}_{-5}$ & $5^{+10}_{-7}$ & $17^{+6}_{-5}$ & $233^{+46}_{-13}$ & $0.84^{+0.20}_{-0.19}$ & $1.46^{+0.07}_{-0.49}$ & $0.44^{+0.05}_{-0.10}$ & $0.19^{+0.01}_{-0.04}$ \\ \hline
 \end{tabular}
 }
\end{table}

Как отмечалось выше, по кривым блеска сверхновой, наблюдаемым в различных изображениях, можно оценить относительную временную задержку. Кривые блеска SN Refsdal аппроксимировались кубическими сплайнами и различными полиномами (\cite{rodney2016}), однако, как отмечают авторы, использование существующей библиотеки шаблонов кривых блеска не позволяет воспроизвести все особенности кривой блеска SN Refsdal, поэтому их результаты требуют уточнения. В работе \cite{petrnat2020} построена физическая модель предсверхновой, удовлетворяющая фотометрическим наблюдениям в разных фильтрах. Это позволило уточнить значения временных запаздываний и коэффициентов усиления между изображениями SN Refsdal. Как и для моделей линз, представленных выше, предполагается, что $\Delta t$ и $\mu$ независимы и не коррелируют друг с другом, их значения приведены в Таблице \ref{tab:dtdmpetrnat}.

\begin{table}[H]
 \caption{Временные задержки (в днях) и относительные усиления между изображениями SN Refsdal, полученные из моделирования предсверхновой (\cite{petrnat2020}).}
 \label{tab:dtdmpetrnat}
 \centering
 \begin{tabular}{ | c | c | c | c | c | c | c | c |}
    \hline
    $\Delta t_{21}$ & $\Delta t_{31}$ & $\Delta t_{41}$ & $\Delta t_{X1}$ & $\mu_{21}$ & $\mu_{31}$ & $\mu_{41}$ & $\mu_{X1}$ \\ \hline
    $9.5^{+2.6}_{-2.7}$ & $4.2^{+2.3}_{-2.3}$ & $30.0^{+7.8}_{-8.2}$ & $340^{+43}_{-52}$ &  $1.14^{+0.02}_{-0.02}$ & $1.01^{+0.02}_{-0.02}$ & $0.35^{+0.016}_{-0.015}$ & $0.24^{+0.12}_{-0.07}$  \\ \hline
 \end{tabular}
\end{table}

%https://arxiv.org/pdf/2007.04106.pdf

% ноутбук по вычислению описанного выше
% https://colab.research.google.com/drive/1euY9KrqhSkNs0WtDz0o3tMGa5Wg6K3vZ#scrollTo=qG9K8eqFDprF