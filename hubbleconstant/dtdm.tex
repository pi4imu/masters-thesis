Для определения постоянной Хаббла необходимы не только измерения временных задержек между различными изображениями, но и информация о гравитационном потенциале скопления, играющего роль линзы. На текущий момент представлен ряд моделей распределения вещества в скоплении галактик MACS J1149+2223, выступающего в роли линзы, разработанных разными коллективами авторов, и приведены теоретические предсказания временных задержек $\Delta t$ и коэффициентов усиления $\mu$ изображений S2, S3, S4, SX и SY относительно изображения S1 (\cite{treu2016}). Изображение SY далее не учитывается, так как оно не наблюдалось. В данной работе рассматриваются модели ‘Gri-g’, ‘Ogu-g’, ’Ogu-a’, ’Sha-g’, and ’Sha-a’. Предполагается, что $\Delta t$ и $\mu$ распределены нормально и не скоррелированы друг с другом, их средние значения и статистические погрешности приведены в

(\cite{treu2016}, (см. Табл. 6)) - мб вставить таблицу? \nl{да, + надо пояснить, откуда такие названия моделей} 

Как отмечалось выше, по кривым блеска сверхновой, наблюдаемым в различных изображениях, можно оценить относительную временную задержку.


%На данный момент опубликованы кривые блеска первой гравитационно-линзированной сверхновой с коллапсирующим ядром SN Refsdal, для которой наблюдались пять ее изображений (\cite{rodney2016}). На основе этих данных проведено детальное моделирование кривых блеска, оценены параметры предсверхновой, уточнены временные запаздывания между изображениями и определены относительные коэффициенты усилений (\cite{petrnat2020}). 

 


Например, в (\cite{rodney2016}) кривые блеска аппроксимировались шаблонами, а также кубическими сплайнами. Отметим, что кривые блеска SN Refsdal плохо описываются при помощи шаблонов ранее наблюдавшихся сверхновых, поэтому в (\cite{petrnat2020}) построена физическая модели предсверхновой, удовлетворяющая фотометрическим наблюдениям в разных фильтрах. Это позволило уточнить значения временных запаздываний и коэффициентов усиления между изображениями SN Refsdal. Как и для моделей линз, предполагается, что $\Delta t$ и $\mu$ независимы и не коррелируют.

%Запланированные результаты имеют высокую научную значимость, т.к. на данный момент опубликованы лишь предварительные оценки временных запаздываний между всеми изображениями сверхновой Рефсдала в работе Родни и др. 2016. При этом авторы публикации подчеркивают, что использование существующей библиотеки шаблонов кривых блеска не позволяет воспроизвести все особенности кривой блеска SN Refsdal, поэтому их результаты требуют уточнения. Построение физической модели предсверхновой, удовлетворяющее фотометрическим наблюдениям в разных фильтрах, позволит уточнить значения временных запаздываний и коэффициентов усиления, что необходимо для оценки постоянной Хаббла, а также послужит критическим тестом разных моделей линзы, представленных в литературе. 
