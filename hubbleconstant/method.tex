Для оценки постоянной Хаббла используется подход, предложенный в работе \cite{vegaferrero}. Так как при построении моделей линзы и вычислении $\Delta t$ (см. Табл. \ref{tab:dtdm}) предполагалось, что постоянная Хаббла равна $H_0^{fid} = 70$ км с$^{-1}$ Мпк$^{-1}$, то значения временных задержек для каждой модели необходимо перемасштабировать таким образом, чтобы предсказанные значения наиболее точно совпали с наблюдаемыми. Из выражения \eqref{eq:ang_dia_dist} видно, что при прочих равных соотношение $H_0\Delta t$ остаётся постоянным. Перемасштабирование выполняется следующим образом:

\begin{equation}\label{eq:rescale}
p_{lens}(\Delta t, \mu | H_0, G) = p_{lens}(\frac{H_0^{fid}}{H_0} \Delta t, \mu | H_0^{fid}, G), 
\end{equation}

\noindent где $p_{lens}$  - вероятность того, что в отдельно взятой модели постоянная Хаббла примет значение $H_0$ при заданных $\Delta t$, $\mu$ между выбранными изображениями, а также распределении $G$ массы в гравитационной линзе. Вероятность $P( H_0 |D)$ того, что постоянная Хаббла примет некоторое значение $H_0$ при заданных данных $D$, равна произведению априорной вероятности $P(H_0)$ на произведение распределений вероятности $p_{lens}(\Delta t, \mu| H0, G)$ для временных задержек и усилений в отдельно взятой модели линзы (см Табл. \ref{tab:dtdm}) и $p_{obs}(\Delta t, \mu)$ в “наблюдаемых” \ данных (см. Табл. \ref{tab:dtdmpetrnat}): 

\begin{equation}\label{eq:probability}
P( H_0 |D) \sim P(H_0) P( D | H_0 ) \sim P(H0) \int d \Delta t \ d \mu \ p_{lens}(\Delta t, \mu| H_0, G) \ p_{obs}(\Delta t, \mu).
\end{equation}
Вероятность в знаменателе формулы Байеса опускается, так как она постоянна для всех моделей.