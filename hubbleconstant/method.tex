Для оценки постоянной Хаббла используется подход, предложенный в (\cite{vegaferrero}). Так как при построении моделей линзы (\cite{treu2016} (см. Табл. 6)) предполагалось, что постоянная Хаббла равна $H_0^{fid} = 70$ км с$^{-1}$ Мпк$^{-1}$, то значения временных задержек для каждой модели необходимо перемасштабировать таким образом, чтобы предсказанные значения наиболее точно совпали с наблюдаемыми. Логика этой операции проста: из выражения \eqref{ang_dia_dist} видно, что при прочих равных соотношение $H_0\Delta t$ остаётся постоянным. Перемасштабирование выполняется следующим образом:

\begin{equation}
p_{lens}(\Delta t, \mu | H_0, G) = p_{lens}(\frac{H_0^{fid}}{H_0} \Delta t, \mu | H_0^{fid}, G), 
\end{equation}

где $p_{lens}$  - вероятность того, что в отдельно взятой модели постоянная Хаббла примет значение $H_0$ при заданных $\Delta t$, $\mu$ между выбранными изображениями, а также распределении $G$ массы в гравитационной линзе. Далее применяется байесовский анализ: вероятность $P( H_0 |D)$ того, что постоянная Хаббла примет некоторое значение $H_0$ при заданных данных $D$, равна произведению априорной вероятности $P(H_0)$ на произведение распределений вероятности $p_{lens}(\Delta t, \mu| H0, G)$ для временных задержек и усилений в отдельно взятой модели линзы из (\cite{treu2016}) и $p_obs(\Delta t, \mu)$ в “наблюдаемых” данных (\cite{petrnat2020}): 

\begin{equation}
P( H_0 |D) \sim P(H_0) P( D | H_0 ) \sim P(H0) \int d \Delta t \ d \mu \ p_{lens}(\Delta t, \mu| H_0, G) \ p_{obs}(\Delta t, \mu).
\end{equation}
Вероятность в знаменателе формулы Байеса опускается, так как она постоянна для всех моделей.