\begin{figure}[H]
    \centering
	\includegraphics[scale=0.99]{pics/snrefsdal.png}
	\caption{\textit{Слева}: схематичное изображение хода лучей от SN Refsdal к  наблюдателю (источник: Space Telescope Science Institute). \textit{Справа}: изображения SN Refsdal (\cite{treu2015}) \label{fig:snrefsdalfig}}
\end{figure}

Открытая в 2014 году в ходе обзора на телескопе \textit{Hubble Space Telescope} (\cite{kelly2014}) сверхновая SN Refsdal находится в рукаве спиральной галактики на красном смещении $z_S= 1.49$, которая линзируется скоплением галактик MACSJ1149.6+2223, находящимся на $z_L = 0.54$, \nl{выбери одну систему обозначений и ее используй на протяжении всего диплома. В пред главе у тебя параметры линзы были $D_d$, $z_d$ и так далее}
таким образом, что существуют сразу три изображения родительской спиральной галактики (изображения 1.1, 1.2 и 1.3 на правой панели Рисунка \ref{fig:snrefsdalfig}). При этом в изображении 1.1 сверхновая дополнительно линзируется эллиптической галактикой скопления таким образом, что формируются четыре её изображения S1-S4 (правая нижняя панель Рисунка \ref{fig:snrefsdalfig}), расположенных в виде “креста Эйнштейна”. Изображение сверхновой SX (средняя панель справа) интересно тем, что его появление в 2015 году было предсказано с высокой точностью \nl{ссылки}. Согласно теоретическим оценкам, SY (правая верхняя панель) - это изображение сверхновой, которое “вспыхнуло” ~20 лет назад и уже угасло  (\cite{kelly2014}, \cite{treu2016}).

%https://arxiv.org/pdf/1411.6009.pdf - appearance of refsdal