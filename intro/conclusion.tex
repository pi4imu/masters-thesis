В данной работе представлен алгоритм извлечения оценки постоянной Хаббла на основе данных по всем 5 изображениям сверхновой SN Refsdal (S1-S4, SX). Исходя из сравнения прогнозов временных задержек и коэффициентов усиления, рассчитанных в различных моделях гравитационного потенциала линзы-галактики и масштабированных таким образом, чтобы они совпадали с наблюдаемыми величинами, вычислено значение постоянной Хаббла, наиболее точно удовлетворяющее и модельным, и наблюдательным данным: $H_0=68.1^{+12.9}_{-10.0}.$  Значение было рассчитано по изображениям SX и S2 сверхновой SN Refsdal. Существенно улучшить оценку величины $H_0$ поможет увеличение числа рассматриваемых гравитационно линзированных систем в целом и более детальная кривая блеска в изображении SX в частности. Полученные результаты также могут послужить независимым тестом различных  моделей распределения масс в линзе-галактике. %Оценка могла бы быть существенно выше, если бы была доступна более детальная кривая блеска для SX.

При анализе гравитационно линзированных сверхновых с наблюдаемыми множественными изображениями одним из важнейших систематических эффектов является микролинзирование звездами галактики-линзы, которое может существенным образом изменять наблюдаемые кривые блеска. Звезды в галактике образуют богатую сеть каустик, что приводит к тому, что наблюдаемые кривые блеска при расширении сверхновых испытывают зависящие от времени усиления/ослабления, уникальные для каждого изображения. Проведено моделирование влияния микролинзирования на кривые блеска сверхновых с коллапсирующим ядром на примере SN Refsdal. Разработан комплекс вычислительных программ, позволяющий учесть реалистичное распределение яркости, а также неточечность источника излучения. На основе проведенного анализа построено распределение вероятности усиления вследствие микролинзирования в звездных величинах и оценено возможное влияние на точность оценок временных запаздываний между изображениями, и, как следствие, на точность определения постоянной Хаббла.

Получена оценка неопределённости, вносимой в определение временных задержек между изображениями S1 и S2 SN Refsdal только микролинзированием в условиях идеального эксперимента. Погрешность составляет 1-3 дня. Таким образом, для определения постоянной Хаббла по гравитационно линзированым сверхновым необходимо использовать системы с временными задержками, составляющими десятки дней. В случае с SN Refsdal это означает, что значение $H_0$ наиболее надежно определяется по паре изображений S2-SX, а анализ изображений S1-S4 позволяет лучше ограничить модель линзы. 

Показано, что даже в условиях идеального эксперимента точность измерения усилений изображений составляет порядка 1 зв.вел. Стоит отметить, что значения макроусиления в моделях линз выводятся, как правило, без учёта микролинзирования. Анализ, проведённый в данной работе, показывает, что, микролинзирование может вносить существенный разброс в определение макроусилений.



%Целью данной работы была оценка возможного вклад микролинзирования при условии проведения идеального эксперимента, в котором гравитационно линзированная система наблюдается практически непрерывно, а ошибки наблюдений сведены к нулю. %Планируется проведение более приближенного к реальности эксперимента, когда учитывается реалистичная частота наблюдений и характеристики настоящих телескопов.
 
