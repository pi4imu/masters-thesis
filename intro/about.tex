\textcolor{red}{1.1 и 1.2 надо объединить в один слитный кусок текста, где-то поменяв местами абзацы}
В настоящее время значения основных космологических параметров известны с очень высокой точностью. В рамках современной стандартной космологической модели $\Lambda$CDM некоторые из них определены с точностью один процент или лучше. При этом в данный момент наблюдается систематическое расхождение в значениях постоянной Хаббла, измеренных по ранней и по поздней Вселенной. Так, по данным обcерватории \textit{Planck} из наблюдений реликтового излучения величина $H_0$ составляет $67.4 \pm 0.5$ км/с/Мпк (\cite{planck2018}). В то же время из наблюдений сверхновых типа Ia, так называемых «стандартных свечей», получено не зависящее от используемой космологической модели значение постоянной Хаббла, равное $74.03 \pm 1.42$ км/с/Мпк (\cite{riess2019}). Такое несоответствие (эти значения не согласуются друг с другом на уровне значимости примерно $4\sigma$) может свидетельствовать как о наличии неучтенных или неизвестных систематических эффектов, так и о новой физике, выходящей за рамки модели $\Lambda$CDM. Для понимания причин этого расхождения необходимо привлечение независимых подходов, способных также с высокой точностью определять фундаментальные космологические параметры, одним из которых является использование наблюдений гравитационно линзированных сверхновых. Как уже было сказано выше, в обозримом будущем ожидается обнаружение большого количества таких объектов.  Точность оценки космологических параметров из наблюдений гравитационно линзированных систем зависит как от надежности моделирования гравитационного потенциала линзы, так и от точности определения временных запаздываний между изображениями источника. Поэтому новой и чрезвычайно актуальной становится задача учета и минимизации возможных систематических эффектов, которые способны существенным образом изменять кривые блеска сверхновых и влиять на точность определения космологических параметров. Одним из таких эффектов является микролинзирование - гравитацинное линзирование на отдельных звёздах в галактике-линзе, уникальное для каждого изображения источника.  

Большое количество публикаций посвящено влиянию микролинзирования на кривые блеска гравитационно линзированных квазаров (одна из первых работ: \cite{changrefsdal1979}), рассмотрена зависимость значимости этого эффекта в зависимости от размера и профиля яркости квазара (\cite{sizeiseverything}), разработаны программы для численного моделирования микролинзирования квазаров (\cite{wambsganss1992}). Однако переменность их блеска непредсказуема, и для точных измерений временных задержек между изображениями необходимо накомить многолетний массив данных. По некоторым оценкам, для ощутимой компенсации ошибок, вызванных влиянием микролинзирования на кривые блеска гравитационно линзированных квазаров, необходимо проводить наблюдения в течение 20 лет (\cite{20years}).

В отличие от квазаров, кривые блеска сверхновых имеют чётко выраженный пик, а их наблюдения занимают сравнительно небольшие времена, что значительно упрощает измерения временных задержек между изображениями и, как следствие, постоянной Хаббла. Тем не менее, микролинзирование может вносить существенные искажения в форму кривых блеска гравитационно линзированных сверхновых (\cite{doblerkeeton2006}). Методика минимизации влияния микролинзирования на ошибки временных задержек предложена, в основном, для сверхновых типа Ia (\cite{moresuyu2017}, \cite{goldstein2018}, \cite{foxleymarrable2018}, \cite{bonvin2019}, HOLISMOKES-III: \cite{holismokesIII}). В упомянутых работах, помимо прочего, показано, что микролинзирование на ранних стадиях расширения сверхновой может быть хроматичным, то есть зависящим от длины волны, тогда как в целом гравитационное линзирование ахроматично. Кроме того, кривые блеска сверхновых типа Ia поддаются стандартизации (\cite{1a_standart}). Однако для сверхновых с коллапсирующим ядром, к которым относится SN Refsdal (\cite{kelly2016}), необходимо разработать новые алгоритмы (HOLISMOKES-V: \cite{holismokesV}), так как их кривые блеска демонстрируют большое разнообразие, что не позволяет их использовать как стандартные свечи напрямую. Разработаны программы для симуляции наблюдений линзированных сверхновых (\cite{pierelrodney2019}). Актуально это в том числе и для разработки тактики наблюдений грядущих обзоров (\cite{hubersuyu2019}). Детальное моделирование кривых блеска сверхновых II типа может дать ценную информацию не только о строении предсверхновых, в том числе находящихся на больших красных смещениях, но и обеспечить высокую точность измерения временных задержек и определения коэффициента усиления изображений сверхновой в отсутствие микролинзирования.


SN Refsdal - первая обнаруженная линзированная сверхновая со множественными изображениями и измеренными кривыми блеска для каждого изображения - открывает эру практического использования линзированных сверхновых для тестов космологических моделей (\cite{grillo2018}). Более того, её наблюдения предоставили возможность определить физические параметры предсверхновой и детально промоделировать кривые блеска столь удаленной сверхновой ($z = 1.5$), что было бы невозможно в отсутствии линзы. На данный момент опубликованы кривые блеска первой гравитационно-линзированной сверхновой с коллапсирующим ядром SN Refsdal, для которой наблюдались пять ее изображений (\cite{rodney2016}). На основе этих данных проведено детальное моделирование кривых блеска, оценены параметры предсверхновой, уточнены временные запаздывания между изображениями и определены относительные коэффициенты усилений (\cite{petrnat2020}). 

(?) Это в том числе позволяет провести сравнение возможных моделей распределения вещества в скоплении галактик MACS J1149+2223, выступающего в роли линзы. Так как помимо микролинзирования, другим источником возможных систематических ошибок является используемая модель линзы, для уточнения которой необходимо привлекать дополнительные данные, например, оптические измерения дисперсии лучевых скоростей. Для оценки надежности методов определения гравитационного потенциала скоплений на основе оптических данных необходимы большие выборки скоплений галактик, которые стали доступны только в последнее время, что также определяет новизну и новые пути решения поставленной задачи.


