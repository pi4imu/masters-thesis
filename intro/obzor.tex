Гравитационное линзирование -- \nl{это отклонение}  траектории света от прямолинейной в гравитационном поле массивных объектов. Это многогранное по своим свойствам и подробно описанное в литературе явление (см., например, \cite{schneider1992}, \cite{gravlensbook}) является, в том числе, одним из независимых способов измерения ряда космологических констант, в частности, постоянной Хаббла $H_0$, определяющей темп расширения Вселенной в современную эпоху. Важным свойством гравитационного линзирования является возможность формирования нескольких изображений одного и того же источника. Точное измерение временной задержки между кривыми блеска, наблюдаемыми в различных изображениях, позволяет с высокой точностью оценить значение постоянной Хаббла (\cite{timedelaycosmography}), \nl{мы ж ничего не уточнили!!!} уточнению значения которой и посвящена данная работа.


\textcolor{red}{Предлагаю вставить кусок текста вот отсюда https://docs.google.com/document/d/1LeG-XjcpTT6cA9TJIDq61qVxFLF8IR38VrDoK6hkXco/edit?usp=sharing. Он хорош тем, что выверен с точки зрения русского языка}
\nl{В настоящее время активно развиваются методы оценки космологических параметров, в том числе постоянной Хаббла, основанные на анализе наблюдаемых гравитационно-линзированных квазаров. Многолетние наблюдения множественных изображений линзированных галактиками квазаров позволили с хорошей точностью измерить временные запаздывания между изображениями. В рамках проекта “COSMOGRAIL” (COSmological MOnitoring of GRAvitational Lenses) уже более десяти лет проводят регулярные наблюдения кривых блеска примерно тридцати линзированных квазаров и разрабатывают методы определения временных запаздываний с целью определения постоянной Хаббла с точностью ~3\%, что будет уже сравнимо с точностью, полученной из наблюдений обсерватории “Планк”. Однако в ходе выполнения данного проекта выяснилось, что проблема вырождения между гравитационным потенциалом линзы и постоянной Хаббла не позволяет достичь заявленной точности
без привлечения дополнительной информации. В связи с этим был инициирован новый проект H0LiCOW (H0 Lenses in COSMOGRAIL's Wellspring), в рамках которого для анализа были выбраны пять гравитационно-линзированных квазаров, для которых проводятся дополнительные фотометрические и спектроскопические наблюдения с высоким разрешением, в том числе измерения дисперсии скоростей звезд в галактике-линзе. Это позволит получить распределение массы с точностью до нескольких процентов и оценить постоянную Хаббла с точностью <3.5\% (Suyu et al. 2017).   
Важно отметить, что надежные измерения временных запаздываний между изображениями квазаров из наблюдений их кривых блеска требуют длительного (около 10 лет) мониторинга. Это связано с наличием как микролинзирования, так и возможной внутренней переменности квазаров на схожих с микролинзированием масштабах.
В последние годы также развивается направление исследования сверхновых, которые линзированы галактиками или скоплениями галактик.} \textcolor{red}{(поправить/вставить ссылки)} Идея использования линзированных сверхновых для 
%В качестве возможных источников для таких измерений используются гравитационно линзированные квазары. На текущий момент опубликовано большое количество работ, посвящённых измерению при их помощи постоянной Хаббла. Результаты представлены в публикациях коллабораций H0LiCOW (\cite{holicowXIII}), TDCOSMO (\cite{tdcosmoI}), STRIDES (\cite{strides}) и COSMOGRAIL (\cite{cosmograil}). Другим подходящим источником являются линзированные сверхновые, идея использования которых для 
оценки $H_0$ была впервые предложена Сьюром Рефсдалом в 1964 году (\cite{refsdal1964}). На текущий момент известны только две гравитационно линзированные сверхновые со множественными изображениями, SN Refsdal (\cite{kelly2014}) и SN iPTF16geu (\cite{goobar2017}). Однако с запуском в ближайшее время обзора Legacy Survey of Space and Time (LSST) на обсерватории им. Веры Рубин, телескопа \textit{Roman Space Telescope}\footnote{Предыдущий вариант названия -- Wide Field Infrared Survey Telescope (WFIRST).}, а также в ходе обзора Zwicky Transient Facility на Паломарской обсерватории ожидается открытие сотен и даже тысяч таких систем \color[red]{точно тысяч? Про сотни поверю, про тысячи сомневаюсь} (\cite{ogurimarshall}, \cite{goldsteinnugent2017}, \cite{veracrubin}, \cite{rst}). Наличие в обозримом будущем большого объёма наблюдательных данных делает задачу разработки алгоритма анализа гравитационно линзированных сверхновых важной и своевременной. Также предложены алгоритмы для обработки систем с неразрешёнными изображениями (\cite{beitunresolved}). 

Стоит отметить, что для сверхновых типа Ia - «стандартных свечей» - можно оценить болометрическую (интегральную по всему спектру) светимость из независимых от линзирования соображений, а значит, получить абсолютное значение усиления светового потока. Эта информация, недоступная для линзированных квазаров, позволяет уменьшить количество свободных параметров модели линзы и снять определенные вырождения (\cite{falco1985}, \cite{holz2001}, \cite{ogurikawano2003}), а значит, уменьшить ошибки на  оцениваемые космологические параметры. Кроме того, по результатам наблюдений гравитационно линзированных сверхновых типа Ia можно уточнить шкалу расстояний в астрономии (\cite{ddr}).

Гравитационно линзированные сверхновые открывают новые возможности для изучения физики сверхновых, в частности, их звёзд-предшественников. \nl{В принципе, в линзированных сверхновых со множественными изображениями возможно наблюдение момента взрыва сверхновой,} %такие системы позволяют наблюдать взрыв сверхновой с самого начала, 
что ранее не представлялось возможным, учитывая сложность обнаружения сверхновых на ранних фазах и необходимость организации последующих наблюдений. Используя временную задержку между появлением различными изображениями, можно смоделировать линзирующую систему и на основе первого изображения (или нескольких изображений) сверхновой предсказать \nl{и пронаблюдать} появление следующих изображений. Наблюдения на ранней фазе имеют решающее значение для понимания природы предсверхновых (\cite{holismokesI}).