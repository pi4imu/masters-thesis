%В настоящее время значения основных космологических параметров известны с очень высокой точностью. В рамках современной стандартной космологической модели $\Lambda$CDM некоторые из них определены с точностью один процент или лучше. При этом значения постоянной Хаббла $H_0$, которая определяет темп расширения Вселенной на текущую эпоху, измеренные по ранней Вселенной, в частности, по реликтовому излучению, систематически меньше значений, измеренных по поздней Вселенной -- например, по сверхновым типа Ia и гравитационно линзированным квазарам. Эти значения не согласуются друг с другом на уровне значимости примерно $4\sigma$, что может свидетельствовать как о наличии неучтенных или неизвестных систематических эффектов, так и о новой физике, выходящей за рамки модели $\Lambda$CDM.

%Для понимания причин этого расхождения необходимо привлечение независимых подходов, способных также с высокой точностью определять фундаментальные космологические параметры. Одной из таких возможностей является использование наблюдений гравитационно линзированных систем, в частности, линзированных сверхновых. Точность оценки космологических параметров из наблюдений гравитационно линзированных систем зависит как от надежности моделирования гравитационного потенциала линзы, так и от точности определения временных запаздываний между изображениями источника. 

%При анализе гравитационно линзированных сверхновых с множественными изображениями одним из важнейших систематических эффектов является микролинзирование звездами галактики-линзы, которое может существенным образом изменять наблюдаемые кривые блеска, что в свою очередь затрудняет как моделирование кривой блеска сверхновой, так и определение временных запаздываний между изображениями, необходимые для определения космологических параметров. Как следствие, разработка алгоритма, позволяющего учесть данный эффект, является чрезвычайно актуальной и востребованной для тестирования космологической модели. 

\begin{center}
    \Large{\textbf{Аннотация}}
\end{center}

В данной работе представлен алгоритм извлечения оценки постоянной Хаббла на основе данных по всем 5 наблюдаемым изображениям сверхновой SN Refsdal (S1-S4, SX). На основе сравнения прогнозов временных задержек и коэффициентов усиления SN Refsdal, рассчитанных в различных моделях гравитационного потенциала линзы-галактики и масштабированных таким образом, чтобы они совпадали с наблюдаемыми величинами, вычислено значение постоянной Хаббла, наиболее точно удовлетворяющее и модельным, и наблюдательным данным: $H_0=68.1_{-10.0}^{+12.9}$. При анализе гравитационно линзированных сверхновых с наблюдаемыми множественными изображениями одним из важнейших систематических эффектов является микролинзирование звездами галактики-линзы, которое может существенным образом изменять наблюдаемые кривые блеска. Звезды в галактике образуют богатую сеть каустик, что приводит к тому, что наблюдаемые кривые блеска при расширении сверхновых испытывают зависящие от времени усиления/ослабления, уникальные для каждого изображения. Промоделирован вклад микролинзирования в кривые блеска SN Refsdal в модели сверхновой с реалистичным профилем яркости. Получена оценка погрешности, вносимой в определение временных задержек между изображениями S1 и S2 SN Refsdal только микролинзированием в условиях идеального эксперимента: 1-3 дня. Показано, что даже в условиях идеального эксперимента точность измерения усилений составляет порядка 1 зв.вел. 

%Работа содержит ААА страниц основного текста и БББ рисунков. При создании работы был использован ВВВ источник.