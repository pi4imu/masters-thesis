%Несмотря на большой интерес научного сообщества к SN Refsdal для космологических тестов, до настоящего момента не проводились исследования влияния микролинзирования на ее кривые блеска. 

В данной работе проведено моделирование влияния микролинзирования на кривые блеска сверхновых с коллапсирующим ядром на примере сверхновой Рефсдала. Разработан комплекс вычислительных программ, позволяющий учесть реалистичное распределение яркости, а также неточечность источника излучения. На основе проведенного анализа посредством разработанного математического обеспечения построено распределение вероятности усиления вследствие микролинзирования в звездных величинах и оценено возможное влияние на точность оценок временных запаздываний между изображениями, и, как следствие, на точность определения постоянной Хаббла.

%Разработан алгоритм извлечения с высокой точностью оценки постоянной Хаббла на основе данных по всем 5 изображениям сверхновой (S1-S4, SX) в отличие от работ \cite{vegaferrero}, \cite{grillo2018}, где использовано только запаздывание между изображениями SX и S1, т.к. первоначальные оценки временных запаздываний между изображениями S1-S4 (\cite{rodney2016}) имеют недостаточную точность.

